\section{概述}

\subsection{研究背景及意义}

近年来,随着人口红利的消失,各行业的劳动力成本不断攀升,建筑业也不例外。随着建筑业的高速发展,现有劳动力出现供不应求、价格攀升的状态,自从建筑行业的生产力相对于其他制造业有着明显低效,建筑行业早就孕育了建筑机器人的发展进程。另一方面,依靠人力进行建筑装修容易出现良莠不齐建筑,建筑豆腐渣工程的报道层出不穷,利用机器人进行建筑装修,可以尽可能使得标准统一,合格率上升,减少人为偷工减料而出现的豆腐渣工程。在“中国制造2025规划”等国家政策的推动下,各行各业推进在智能设备、智能制造技术上的迭代升级,建筑行业也需要与时俱进。

根据建筑工地的多重程序,其中包括有打桩、抹灰、刷粉、铺墙砖、铺地砖等等,基于建筑机器人在国内还比较罕见,再根据比较建筑工序运用到机器人的开发难易程度和传统建筑工序中人力分布程度,创业团队选择了铺地砖这一程序进行优化和改良,开发机器人缓冲劳动力的缺陷。

在团队分工中,本课题的主要任务是基于OpenCV的工程图数据提取及其在UWB定位系统的应用。其中,OpenCV\footnote{\url{https://opencv.org/}}(Open Source Computer Vision Library)是一个跨平台的开源计算机视觉库,其应用领域包括人脸识别、手势识别、图像分割等,而在本课题要运用到OpenCV的主要功能是图像处理,由于建筑行业普遍使用的工程图是.dwg格式文件,鉴于.dwg格式文件不开源,并且没有Python的第三方开发包,加上提取图纸数据困难,所以团队采取了将.dwg工程图纸转成同规格大小的JPG图像格式文件,再利用OpenCV库进行对JPG工程图像进行数据提取的办法。由于计算机视觉在现今是一个热门且发展迅速的领域,加上OpenCV开发文档和论坛社区较为完善,所以利用OpenCV对JPG工程图像数据提取,可以省去破解不开源的.dwg文件这一程序,能够极大地节省了开发周期,并且适用性广。

其次,超宽带(Ultra-wideband,简称UWB),是一种具有低耗电与高速传输的无线个人局域网通讯技术。本课题以超宽带室内定位技术为基础,搭建室内铺砖机器人定位系统,将CAD工程图纸数据进行提取,导入铺砖机器人定位系统中,建立室内施工环境全局坐标系,对机器人的当前位置、移动趋势与路径进行远程监控,创新的利用目前精度较高的室内定位方案运用在建筑行业当中。


\subsection{同类机器人在国内外的发展现状}
来自荷兰RoadPrinter公司的一种大型半自动铺路砖机Tiger-Stone,该机器能够半自动铺设耐用又美观的砖路。铺路机器分为填料区和铺设区,工人站在填料区进行填料,随着铺砖机器在沙基路面上一点点向前行进,砖块就会因为地心引力自动挤在一块。但是,施工时需要人力将砖块按照有角度的填料槽进行填充。而全自动化的铺砖机器人基本不需要人工辅助,所有程序预先设定,机器人便能高效地进行路面铺设。

早在2014年,国外 Singapore-ETH Centre Future Cities Laboratory and ROB Techologies 的研究人员研发出一款全自动化能铺地板砖的机器人雏形。该初步模型是基于六轴机械臂进行铺贴,通过红外扫描需要铺贴地砖的区域,然后从机器人取下地砖,能精准地将地砖铺到测量好的位置,工作效率是目前人工铺贴的四倍。

目前,国内的大部分房地产公司在转型涉及机器人产业。比如,2018年9月碧桂园计划5年内在机器人领域投入至少800亿元,其中就包括在建筑行业的多个方面研发机器人,而铺贴地板砖就是其中一个方面。


\subsection{本文主要工作}
本文从对建筑工程图的认识展开,讨论了将AutoCAD工程图.dwg文件转换为JPG图像格式文件的原因,并通过OpenCV对JPG工程图进行数据提取和处理,然后将提取出来的工程图数据利用Qt开发框架绘制出室内区域和砖块摆放样式,最后利用UWB室内定位技术将机器人在室内的位置进行实时标记。


\subsection{论文组织结构}
本论文共分为七章,内容如下:

第一章为引言,主要介绍了本论文的研究背景、意义,同类机器人在国内外的发展现状,主要工作及论文的组织结构。

第二章为预备知识,首先对建筑工程图进行简单的认识,其次分析OpenCV在该系统中的应用,并且对OpenCV进行基本的认识,最后简要介绍UWB室内定位系统原理和算法。

第三章为基于OpenCV的工程图数据的数据提取,主要说明了采用OpenCV提取工程图数据的思路,研究了OpenCV在工程图的数据提取和分析各个步骤中的算法分析。

第四章为工程图数据在UWB定位系统的应用,主要研究了根据工程图数据利用Qt\footnote{\url{https://www.qt.io/cn}}绘制室内平面图,然后规划砖块摆放,并将砖块信息返回给机器人,最后通过Qt利用Python和C++混合编程绘制出机器人的定位位置。

第五章为实验结果,阐述了本课题整个设计的流程图和整个系统的在实际应用场景中的表现。

第六章为结论与展望,首先简要总结了本文的一些工作,并对接下来进一步的研究工作做了展望。 