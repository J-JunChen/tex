\newpage


\centerline{\fangsong\bf\zihao{-2}{基于OpenCV的工程图数据提取及其}}

\centerline{\fangsong\bf\zihao{-2}{在UWB定位系统的应用}}

\addcontentsline{toc}{section}{摘要(关键词)}%加入目录

\vskip 1cm

\begin{center}
	\kaishu
	\hspace{2cm}机器人学院软件工程专业 \quad 陈俊杰 
	\vspace{5bp}
	\newline
	学号:201541412127
\end{center}

\vskip 10bp
{
\kaishu	
\hspace{5bp}{\zihao{-4}\textbf{【摘要】}} 
本文介绍了铺砖机器人的研究背景和意义,列举了同类机器人在国内外的发展现状,肯定了铺砖机器人在未来发展的前景。本文从铺砖机器人在实际施工场地中室内定位的需求出发,提出了提取工程图数据和实时定位两个部分的解决方案。第一部分中工程图由常规的DWG格式文件转换到JPG格式图像,再经过OpenCV处理,检测出工程图中的角点,并通过角点定位排序法;第二部分利用排好序的角点,通过Qt绘制出等比例的工程图,并利用砖块对工程图进行栅格化,再将砖块信息反馈给铺砖机器人,最后利用UWB串口返回的数据通过Python和C/C++的混合编程定位机器人在室内的实时位置。


\vskip 10bp

\hspace{5bp} {\zihao{-4}\textbf{【 关键词 】}} 
铺砖机器人; DWG2JPG; PyQt5; Trilateration定位; 混合编程
}
