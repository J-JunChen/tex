\documentclass[a4paper, 11pt]{article}
\usepackage{xeCJK}
\begin{document}

% Example 1
\ldots when Einstein introduced his formulas
\begin{equation}
e = m \cdot c^2 \; ,
\end{equation}
which is at the same time the most widely known
and the least well understood physical formula.


% Example 2
\ldots from which follows Kirchhoff’s current law:
\begin{equation}
\sum_{k=1}^{n} I_k = 0 \; .
\end{equation}
Kirchhoff’s  voltage law can be derived \ldots

\sloppy

\newpage

% Example 3
\ldots which has several advantages.
\begin{equation}
I_D = I_F - I_R
\end{equation}
is the core of a very different transistor model. \ldots
\\
\TeX \\ \today \\ \LaTeXe \\
\LaTeX

daughter-in-law, X-rated\\
pages 13-67\\
yes---or no? \\
https//www.junstitch.edu/$\sim$demo
$0$, $1$ and -1

It's $-30 ^{\circ}\mathrm{C}$
I will soon start to super-conduct.

Not like this ... but like this:\\ New York, Tokyo, China, \ldots

Not shelfful \\ but shelf\mbox{}ful

H\^otel, na\"\i ve ,\\最强
\paragraph{最强第一段}
最强之俊杰
\subparagraph{最强之1.1}

\footnote{foot note text}

Footnots\footnote{This is a footnote.} are often used by people using \LaTeX

\flushleft
\begin{enumerate}
    environments to your taste:
\begin{itemize}
    \item But it might start to look silly.
    \item [-] With a dash.
\end{itemize}
    \item Therefore remember:
    \item 
    \begin{description}
    \item[Stupid] things will not become smart because 
        smart because they are in a list
    \item[Smart] things, though, can be 
    presented beautifully in a list.
        
        
    \end{description}

\end{enumerate}

\begin{center}
    At the centre \\ of the earth
\end{center}

A typographical rule of thumb
for the line length is:
\begin{quote}
    On average, no line should 
    be longer than 66 characters.
\end{quote}
This is why \LaTeX{} pages have
such large borders by default
and also why multicolumn print is
used in newspapers.

\begin{verbatim}
    10 Print "Hello world"
    20 GOTO 10
\end{verbatim}
\begin{verbatim*}
    这是啥
\end{verbatim*}
\verb|like this :-) |

\begin{tabular}{|r|l|c|}
\hline
7C0 & hexadecimal & haha \\ 
3700 & octal & what \\ \cline{2-3}
111110000 & binary & \\
\hline \hline
1984 & decimal \\
\hline
\end{tabular}

\newpage

Add $a$ squared and $b$ squared to get
$c$ squared. Or, using a more mathematical 
approach: $c^{2}=a^{2}+b^{2}$

\TeX{} is pronounced as \\
100 m $^{3}$ of water \\[6pt]
This comes from my passion
\begin{math}\heartsuit\end{math}

\end{document}
